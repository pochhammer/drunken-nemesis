\documentclass[a4paper, twoside]{extarticle}

\usepackage{amsmath, amsfonts, amsthm, amssymb, euler}
\usepackage[left=1.5cm,top=1.5cm,right=1.5cm,bottom=2cm]{geometry}
\usepackage{multicol}

\newenvironment{enumx}{\begin{enumerate} \setlength{\itemsep}{0pt} \setlength{\parskip}{0pt} \setlength{\parsep}{0pt}}{\end{enumerate}}
\newenvironment{itemx}{\begin{itemize} \setlength{\itemsep}{0pt} \setlength{\parskip}{0pt} \setlength{\parsep}{0pt}}{\end{itemize}}

\usepackage[usenames,dvipsnames]{color}
\definecolor{rot}{rgb}{0.9,  0,    0}
\definecolor{blau}{rgb}{0,    0.45, 0.75}
\newcommand{\farber}[1]{{\color{rot} {#1}}}
\newcommand{\farbeb}[1]{{\color{blau} {#1}}}

\usepackage{hyperref}
\hypersetup{colorlinks, citecolor=white, filecolor=niebieski, linkcolor=black, urlcolor=niebieski}

\relpenalty=10000
\binoppenalty=10000

\usepackage{Alegreya}
\usepackage{microtype}

\newcommand{\Q}{\mathbb{Q}}
\newcommand{\Z}{\mathbb{Z}}
\newcommand{\C}{\mathbb{C}}
\newcommand{\n}{\|\cdot\|}

\newtheorem{mojetw}{Twierdzenie}

\usepackage[polish]{babel}
\usepackage[utf8]{inputenc}
\usepackage[T1]{fontenc}
\selectlanguage{polish}

\author{Leon}
\title{Analiza $p$-adyczna}

\begin{document} 
\maketitle
\tableofcontents

\section{Liczby $p$-adyczne}
\begin{mojetw}[Ostrowski]
Każda nietrywialna wartość bezwzględna na $\Q$ jest równoważna z jedną z wartości bezwzględnych $\|\cdot\|_p$, gdzie $p$ jest l. pierwszą lub $p = \infty$.
\end{mojetw}

\begin{proof}
Niech $\n$ będzie nietrywialną normą na $\Q$.
Rozważmy dwa przypadki.
Pierwszy -- niech $\n$ będzie archimedesowa.
Celem jest pokazanie, że odpowiada jej $\n_\infty$.
Weźmy więc najmniejsze dodatnie całkowite $n_0$, że $|n_0| > 1$.
Wtedy $|n_0| = n_0^\alpha$ dla pewnej $\alpha > 0$.
Wystarczy uzasadnić, dlaczego $|x| = |x|_\infty^\alpha$ dla każdej $x \in \Q$, a właściwie tylko dla $x \in \Z_{>0}$ (ponieważ norma jest multiplikatywna).
Dowolną liczbę $n$ można zapisać w systemie o podstawie $n_0$: $n = a_0 + a_1 n_0 + \dots + a_kn_0^k$, gdzie $a_k \neq 0$ i $0 \le a_i \le n_0-1$.
\begin{align*}
|n| & = \left|\sum_{i=0}^k a_in_0^i\right| \le \sum_{i=0}^k \left|a_i\right| n_0^{i \alpha} \le \sum_{i=0}^k n_0^{i \alpha} = n_0^{k \alpha} \sum_{i = 0}^k n_0^{-i \alpha} \le n_0^{k \alpha} \sum_{i = 0}^\infty n_0^{-i \alpha} = n_0^{k \alpha} \frac{n_0^\alpha}{n_0^\alpha - 1} = C n_0^{k \alpha}
\end{align*}

Pokazaliśmy więc, że $|n| \le Cn_0^{k \alpha} \le C n^\alpha$.
Skoro jest to prawda dla każdego $n$, to w szczególności dla liczb postaci $n^N$ (ponieważ $C$ nie zależy od $n$): $|n| \le C^{1/n}n^\alpha$.
Przechodząc z $N$ do granicy dostajemy $C^{1/n} \to 1$ i $|n| \le n^\alpha$.
Teraz trzeba pokazać nierówność w drugą stronę.
Skorzystamy jeszcze raz z rozwinięcia przy podstawie $n_0$.
Skoro $n_0^{k+1} > n \ge n_0^k$, to zachodzi
\[n_0^{(k+1) \alpha} = |n_0^{k+1}| = |n+n_0^{k+1} - n| \le |n| + |n_0^{k+1} - n|, \text{ więc } |n| \ge n_0^{(k+1)\alpha} - |n_0^{k+1} -n| \ge n_0^{(k+1)\alpha} - (n_0^{k+1} -n)^\alpha\]

Skorzystaliśmy tutaj z nierówności udowodnionej wyżej.
Wiemy, że $n \ge n_0^k$, więc prawdą jest, że
\begin{align*}
|n| & \ge n_0^{(k+1)\alpha} - (n_0^{k+1} - n_0^k)^\alpha = n_0^{(k+1) \alpha} \left(1 - \left(1 - \frac{1}{n_0}\right)^\alpha\right) = C' n_0^{(k+1)\alpha} > C' n^\alpha.
\end{align*}

Teraz $C' = 1- (1-1/n_0)^\alpha)$ nie zależy od $n$, jest dodatnia i analogicznie do poprzedniej sytuacji możemy pokazać prawdziwość $|n| \ge n^\alpha$.
Wnioskujemy stąd, że $|n| = n^\alpha$ i $\n$ jest równoważna ze zwykłą wartością bezwzględną.


Załóżmy teraz, że $\n$ jest niearchimedesowa.
Wtedy $\|n\| \le 1$ dla całkowitych $n$.
Ponieważ $\n$ jest nietrywialna, musi istnieć najmniejsza l. całkowita $n_0$, że $\|n_0\| < 1$.
Zauważmy po pierwsze, że $n_0$ musi być l. pierwszą: gdyby zachodziło $n_0 = a \cdot b$ dla $1 < a,b < n_0$, to $|a| = |b| = 1$ i $|n_0| < 1$ (z minimalności $n_0$) prowadziłoby do sprzeczności.
Chcemy pokazać, że $\n$ jest równoważna z normą $p$-adyczną, gdzie $p := n_0$.
W następnym kroku uzasadnimy, że jeżeli $n \in \Z$ nie jest podzielna przez $p$, to $|n| = 1$.
Dzieląc $n$ przez $p$ z resztą dostajemy $n = rp + s$ dla $0 < s < p$.
Z minimalności $p$ wynika $|s| = 1$, zaś z $|r| \le 1$ ($\n$ jest niearchimedesowa) i $|p| < 1$: $|rp| < 1$.
,,Wszystkie trójkąty są równoramienne'', więc $|n| = 1$.
Teraz wystarczy zauważyć, że dla $n \in \Z$ zapisanej jako $n = p^v n'$ z $p \nmid n'$ zachodzi $|n| = |p|^v |n'| = |p|^v = c^{-v}$, gdzie $c = |p|^{-1} > 1$, co kończy dowód.
\end{proof}

\section{Analiza w $\Q_p$}

\section{Analiza w $\C_p$}
\end{document}